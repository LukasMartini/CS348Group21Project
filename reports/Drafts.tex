\documentclass{article}
\usepackage{graphicx} % Required for inserting images
\usepackage[margin=1in]{geometry} % Set margins
\usepackage{hyperref} % For links
\usepackage{verbatim}

%%% "Set and Forgets"
\usepackage[utf8]{inputenc}
\usepackage[T1]{fontenc}
\usepackage{lmodern}

%%% For better newlines
\usepackage{parskip}

%%% Define a shorthand function to force paragraph indents.
\newcommand{\forceindent}{\parindent=0.7cm\hangindent=0.7cm}

\begin{document}

\section*{Application Features}
    \subsection*{R6 - Content Search}

        One of the core features of our application is the ability to search for and view information about any given hand that a user has played. While conceptually simple, the implementation 
        will make use of every part of our technology stack. To find a played hand, its unique identifier and a player name must be given by the user in our web interface. 
        From there, using account data, we can query our database for results from all of the relevant tables (player\textunderscore action, player\textunderscore cards, board\textunderscore cards) 
        using predicates such that only data associated with the desired user is selected. This data is then sent back to the front-end for viewing. If no player name is given, it will default to the
        user searching the database.

        An example of how this feature will look can be seen in our database-driven application submitted with Milestone 1.

        The query template is as follows:
        \begin{verbatim}
PREPARE handData AS (SELECT poker_hand.id, p.name, poker_hand.total_pot, 
                            player_cards.hole_card1, player_cards.hole_card2,
                            player_action.hand_id, player_action.amount,
                            board_cards.flop_card1, board_cards.flop_card2, board_cards.flop_card3,
                            board_cards.turn_card, board_cards.river_card
                    FROM poker_hand, (SELECT * FROM player WHERE player.name = $2) p, 
                         player_action, player_cards, board_cards
                    WHERE poker_hand.id = $1 AND player_action.player_id = p.id 
                            AND player_action.hand_id = $1
                            AND player_cards.hand_id = $1 AND player_cards.player_id = p.id
                            AND board_cards.hand_id = $1
);
        \end{verbatim}
        where the selected data can only come from rows where the inputted hand\textunderscore id (1) and player name (2) match with the queried tables.
        Note that temporary table 'p' is to translate a player's name to its associated player\textunderscore id. We do this to allow us the flexibility to 
        implement searching by user.

        If we use the data below in a sample query,
        \begin{verbatim}
poker_hand: 1, 1, 1, Cash, 0.1, 0.1, CAD, 0.1, 0.1, January 8 04:05:06 1999 PST, Gaussia, 6

player: 1, Ted

player_action: 1, 1, 1, 1, Folds, 0.01

player_cards: 1, 1, 1, AC, AS, 1, 0.1

board_cards: 1, 1, AC, 2C, 3C, 4C, TC
        \end{verbatim}
        We expect the output to be
        \begin{verbatim}
[(1, 'Ted', Decimal('0.10'), 'AC', 'AS', 1, Decimal('0.01'), 'AC', '2C', '3C', '4C', 'TC')]
        \end{verbatim}


\end{document}